\documentclass[12pt, letterpaper]{article}
\usepackage[utf8]{inputenc}
\usepackage{graphicx}
\usepackage{amsmath}
\usepackage{parskip}
\usepackage{tikz}
\usepackage{hyperref}

\newcommand{\externalLink}[2]{\emph{\underline{\href{#1}{#2}}}}


\title{Notes on Ordinary Differential Equations}
\author{Aaron Pierce}
\date{Spring 2021} % to remove date from \maketitle
\begin{document}

\maketitle

\tableofcontents

\newpage

\section{The Forest View}
Differential equations inherently involve changes.
Their most straightforward applications are to model physical phenomena. 
For example, it's easy to state that a falling apple accelerates at -9.81 meters per second,
however, it's not so easy to ask how many meters that apple has fallen after, say, 3 seconds.

If we define the position of the apple to be $f(t)$, we know that $f''(t) = -g$,
as acceleration is the second derivative of position.

If we take two integrals of that acceleration function, we arrive at the position function,
and we've derived the free fall equation without much work.

Differential Equations are a tool to solve problems with that general shape to them.
It's easy to state how something changes, but not so easy to figure out the exact
value of that something.

In the free fall case, it was unusually easy to state how the system changes, as it was a constant,
but some systems are not as straightforward.

For example, if you have some money in an account accruing 6\% yearly interest,
and you contribute another thousand a year,
then your rate of change is $A'(y) = 0.06A(y) + 1000$.
Where $A$ is how much money you have and $y$ is the number of years that have passed.
Meaning that every year you gain 6\% of however much money you have ($0.06A(y)$),
and then you add a thousand dollars ($+1000$).

So how do you solve this?
Before, we just integrated the derivative function, but now it has
a term that doesn't have a meaningful intergral! $\int A(y) \, dy$ is meaningless to us right now,
so we can't just take $\int A'(y) \, dy$, because $A'(y)$ is itself some function of $A(y)$.

The forest view of Differential Equations (at least from what my class looks like), is to
develop a toolbox of techniques to tackle various forms of equations like these, where
finding solutions to the equations is not obvious, easy, or even possible.
The following notes will be divided into sections by the types of Differential Equations,
and will detail how you would solve such an equation.

Truthfully, I think that sounds woefully boring. This is what we have WolframAlpha for, after all.
But I will blindly believe the masters before me who seem very steadfast that
learning these techniques is a productive use of time.
I'm not in much of a position to argue otherwise.


\section{The First Order Differential Equation}
The order of a Differential Equation is the highest derivative in the function.
So a function that involves first, second, and third derivatives is a third order function.
First order means that only $y$ and $y'$ are legal to be used, along with any functions of $x$ in there.

Speaking of, it's worth establishing some notation.
The variables in all of these equations are assumed to be of $x$,
and all functions are called $y$.
So $y$ is exactly the same was writing $y(x)$, and $y'$ or $y'''$ denote derivaties of $y(x)$.
And derivatives can also be expressed as $\frac{dy}{dx}$, which will be useful in about 2 paragraphs time.

\subsection{Separable Equations}
I usually spend a lot of time trying to gain understandings or intution for 
just about any mathematical concept I learn. However,
separable equations are so darn convienient, and so believable too,
that I'm inclined to only mention their existance, instead of justifying it.

If you have a derivative that looks like $\frac{dy}{dx} = \frac{Q(x)}{R(y)}$,
then you can multiply off the $R(y)$ and the $dx$ to have $R(y) dy = Q(x) dx$,
and you can slap an integral sign on both sides, $\int R(y) \,dy = \int Q(x) \,dx$,
and assuming the functions have integrals, you achieve some function of y equaling some function of x,
so you can re-arrange it and find a solution.

In general, if you can get all the $y$ terms on one side, and all the $x$ terms on the other,
separable equations will make your life easy.

\subsection{Linear Differential Equations}
This one has a neat trick that's actually pretty fun, so we'll take more time here.

If your equation is of the form $y' + p(x)y = q(x)$, we run into the problem from a few sections earlier.
We can't just rearrange the equation and take the integral of $q(x) - p(x)y$, because we can't do anything with $\int p(x)y$.
We don't even know what $y$ is, let alone its integral.

However, there's a neat little trick we can do.
$y' + p(x)y$ looks a little like the product rule.
there's a y' term added to a y term with some other function,
so we're not exactly there, but we're close.
If this term is a product rule, though, then we can find some other function
that has a derivative equal to $y' + p(x)y$. What is that function?
We need to do some tinkering first. If we take $(a(x)y)'$, its derivative is
$a(x)y' + a'(x)y$. That's almost $y' + p(x)y$, just so long as $a(x) = 1$ and $a'(x) = p(x)$.
That's a serious problem, though, because the derivative of 1 is 0, which is probably not equal to $p(x)$

$(ye^{\int P(x) \, dx})' = y'e^{\int P(x) \, dx} + P(x)ye^{\int P(x) \, dx}$
\end{document}